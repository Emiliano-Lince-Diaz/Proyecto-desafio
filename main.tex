\documentclass{article}
\usepackage[utf8]{inputenc}
\usepackage[spanish]{babel}
\usepackage{listings}
\usepackage{graphicx}
\graphicspath{ {images/} }
\usepackage{cite}

\begin{document}

\begin{titlepage}
    \begin{center}
        \vspace*{1cm}
            
        \Huge
        \textbf{Desafío de las Tarjetas (Calistenia)}
            
        \vspace{0.5cm}
        \LARGE
        Informática 2
            
        \vspace{1.5cm}
            
        \textbf{Emiliano De Jesus Lince Diaz}
            
        \vfill
            
        \vspace{0.8cm}
            
        \Large
        Despartamento de Ingeniería Electrónica y Telecomunicaciones\\
        Universidad de Antioquia\\
        Medellín\\
        Marzo de 2021
            
    \end{center}
\end{titlepage}

\tableofcontents
\newpage
\section{Sección introductoria}\label{intro}
Este es un desafio que se puede desarrollar con sus familiares o amigos. Disfrutando, aprendiendo y pasando un buen rato con dicahs personas

\section{Sección de contenido} \label{contenido}
Pasos o instrucciones del desafio:\\
\\
Inicio\\
\\
1. levantar la hoja y colocarla al lado de las tarjetas.\\
\\
2. coger las tarjetas.\\
\\
3. de manera que estén las tarjetas verticalmente y sobre la hoja vas a poner el dedo índice sobre las tarjetas y los dedos pulgar y medio a los lados correspondientes de cada uno (en la tarjeta).\\
\\
4. al final con el dedo anular ir separando las tarjetas hasta mantenerlas en equilibrio con un Angulo interno de aproximadamente 45 grados desde la hoja a cada una de las tarjetas.\\
\\
5. si se pasa del Angulo cerrarlas y vuelva a abrirlas.\\
\\
6. soltarlas y alejar la mano .\\
\\
Final\\
\\
Link del video del desafio: https://youtu.be/skcikQMpqws.

\end{lstlisting}

\end{document}
